\section{Wymagania funkcjonalne}
\label{sec:FunctionalRequirements}

\subsection{Wymagania stawiane danym}
\label{sec:TrainingDataRequirements}
\begin{itemize}
    \item Strukturyzacja danych
    \par
    Na dane powinny składać się trójwymiarowe skany mózgu wraz z maskami odpowiadającymi rzeczywistym położeniom lezji na obrazach.
    
    \item Wystarczająca jakość danych
    \par
    Dane winny być dostarczone w odpowiednio dużej oryginalnej rozdzielczości, pozwalającej na sprawną ekstrakcję cech w procesie uczenia. Maski danych powinny być prawidłowe, precyzyjne i dokładne, a lezje powinny być oznaczone przez kompetentnych ekspertów.
    
    \item Normalizacja danych
    \par
    Dane na wejściu winny być unormowane względem siebie, tak, aby potencjalne dane z różnych źródeł nie wykazywały systemowo różnych strukturalnych cech. Różnice strukturalne powinny być aplikowane w procesie wstępnego przetwarzania danych.
    
    \item Wstępne przetwarzanie danych
    \par 
    Dane na wejściu sieci powinny być objęte wstępnym przetwarzaniem danych, rozszerzając tym samym bazę uczącą sieci.

    \item Status legalny danych
    \par
    Dane powinny być pozyskane w sposób legalny, za zgodą właściciela zbiorów danych. Ze względu na newralgiczność danych medycznych zbioru danych medycznych nie wolno udostępniać bez zgody właściciela.
\end{itemize}

\subsection{Wymagania stawiane architekturze sieci i procesowi treningowemu}
\label{sec:NetworkRequirements}
\begin{itemize}
    \item 
    Złożoność architektury
    \par
    Architektura sieci powinna być stworzona tak, aby pozwalała na przeprowadzenie procesu uczenia z wykorzystaniem dostępnego sprzętu.
    \item 
    Odpowiednie moce obliczeniowe
    \par
    Niezbędne jest posiadanie sprzętu oraz wykorzystanie technologii, np. technologii obliczeń równoległych do akceleracji procesu uczenia.
    \item 
    Odpowiednie parametry uczenia sieci
    \par
    Parametry uczenia sieci powinny być dobrane tak, żeby ta uczyła się sprawnie.
    \item 
    Podział na dane uczące i ewaluacyjne
    \par
    Do procesu uczenia niezbędne jest podzielenie zebranego zbioru danych na dane uczące i ewaluacyjne. Dane uczące powinny być mieszane w procesie uczenia.
\end{itemize}

\subsection{Wymagania stawiane wyjściom sieci}
\label{sec:OutputRequirements}
\begin{itemize}
    \item 
    Jakość wyjść
    \par
    Tworzona maska lezji powinna bardzo dokładnie odzwierciedlać rzeczywiste ich położenie. Nie może zaznaczać lezji tam gdzie jej nie ma bądź nie zaznaczać tam gdzie ona w istocie jest.
    \item 
    Minimalizacja fałszywie pozytywnych wyników
    \par
    Rozwiązanie pod żadnym pozorem nie może zaznaczać lezji tam gdzie jej nie ma. Tworzenie przesłanek chorobowych w przypadku osoby zdrowej może okazać się tragiczne w skutkach.
\end{itemize}
