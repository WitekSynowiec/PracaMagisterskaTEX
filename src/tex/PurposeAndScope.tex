\section{Cel i zakres pracy}
\label{sec:PurposeAndScope}
\subsection{Cel pracy}
\label{sec:Purpose}
\par
Celem niniejszej pracy magisterskiej jest zaprojektowanie, nauczenie i opisanie sieci neuronowej służącej do segmentowania lezji stwardnienia rozsianego na obrazach skanów mózgu, a następnie porównanie wpływu doboru parametrów uczenia na końcową jakość rozwiązania, w tym celu przeprowadzając stosowną analizę statystyczną wyjść.       
\subsection{Zakres pracy}
\label{sec:Scope}
\par
W zakres pracy wchodzi analiza następujących zagadnień:
\begin{itemize}
    \item przegląd rozwiązań istniejących na rynku \chapterLink{sec:PurposeAndScope}
    \item zebranie wymagań funkcjonalnych \chapterLink{sec:FunctionalRequirements}
    \item zebranie informacji o metrykach ewaluacyjnych \chapterLink{sec:evaluation}
    \item zebranie informacji o używanych technologiach z zakresu sieci neuronowych, w tym modelach, funkcjach kosztów \chapterLink{sec:ann}
    \item zebranie informacji dotyczących splotowych sieci neutonowych \chapterLink{sec:cnn}
    \item pozyskanie zbioru danych \chapterLink{sec:DatasetSelection}
    \item dobór odpowiednich narzędzi programistycznych \chapterLink{sec:IDESelection}
    \item dobór sprzętu do procesu uczenia \chapterLink{sec:GPUSelection}
    \item dobór odpowiedniego modelu i parametrów sieci \chapterLink{sec:nn-selection}
    \item implementacja modelu sieci neuronowej \chapterLink{sec:Execution}
    \item uczenie sieci neuronowej \chapterLink{sec:Execution}
    \item ewaluacja sieci neuronowej z wykorzystaniem metryk \chapterLink{sec:Tests}
    \item przeprowadzenie testów praktycznych rozwiązania, analiza wyników \chapterLink{sec:Tests}
    \item wnioski dotyczące przedmiotu pracy \chapterLink{sec:Summary}
\end{itemize}