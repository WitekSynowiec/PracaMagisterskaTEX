\begin{center}
\section*{\large\ThesisPL} 
\end{center}

%\lipsum[1-3]

Stwardnienie rozsiane (MS) jest jedną z najpowszechniejszych chorób neurodegeneracyjnych, często prowadzącą do niepełnosprawności. Może spowodować utratę lub poważny uszczerbek na wzroku, równowadze, kontroli mięśni, oraz wpływa na codzienną aktywność i pogorszenie zdrowia psychicznego. Jednym ze sposobów diagnozy i kontroli przebiegu MS jest analiza lezji na obrazach MRI. Oznaczanie lezji jest procesem czasochłonnym, wykorzystuje dużą ilość ludzkich zasobów oraz wyniki procesu są różne, w zależności od osoby która je dokonuje. Postuluje się stworzenie narzędzia, potrafiącego w automatyczny sposób oznaczyć lezje. W tym celu przedstawia się splotową sieć neuronową, służącą realizacji opisanego zadania. Powstało wiele publikacji zajmujących się podobną tematyką, jednak często wyniki były niezadowalające. Problemem w uczeniu sieci neuronowych do segmentacji lezji MS jest bardzo duże niezbalansowanie danych. W niniejszym rozwiązaniu proponuje się wykorzystanie ważonych funkcji kosztów do rozwiązania tego problemu. Przedstawione sieci osiągają dobre rezultaty w problematyce segmentacji lezji MS. Do procesu uczenia wykorzystano dane ISBI oraz OFSEP, składające się ze skanów MRI oraz oznaczeń lezji.\\[7ex]