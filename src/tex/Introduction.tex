\section{Wprowadzenie}
\label{sec:Introduction}

\par
Niniejsza praca dotyczy dynamicznie rozwijającego się obszaru jakim jest implementacja systemów przetwarzania informacji opartych na sieciach neuronowych w diagnostyce medycznej, tym samym zalicza się do nurtu komputerowo wspomaganej diagnostyki medycznej CAD \en{computer aided diagnosis}. Obserwuje się stały wzrost zainteresowania CAD wraz z pojawiającą się dostępnością nowych technik, a także wraz z usprawnianiem oraz rozszerzaniem istniejących metod akwizycji danych medycznych. W ostatnich latach jedną z najpopularniejszych technik, ze względu na rozwój jednostek obliczeniowych, stały się metody oparte na sztucznych sieciach neuronowych, które zrewolucjonizowały całe segmenty rynków. Dziś powszechnie sztuczną inteligencję wykorzystuje się w handlu, w przemyśle motoryzacyjnym i zbrojeniowym, w marketingu, w przemyśle kosmicznym, a nawet w przeprowadzaniu analiz rynkowych. Naturalnie, wiąże się również nadzieje z wykorzystaniem wspomnianych metod w diagnostyce medycznej.
\par
Samo wykorzystanie metod komputerowych w diagnostyce medycznej nie obywało się bez problemów. Jeszcze w latach 60. XX wieku powszechnie uważano, że za pomocą technik komputerowych, w krótkim okresie czasu, zostanie ona w pełni zautomatyzowana, co ze względu na niewystarczające możliwości komputerów oraz brak zaawansowanych technik przetwarzania obrazów zostało podważone. W latach 80. pogląd ten zastąpiono pojęciem komputerowo wspomaganej diagnostyki medycznej CAD, która w założeniu nie tyle miała ją automatyzować, co pomagać w niej diagnostyce \cite{DOI2007198}. Wśród jej niewątpliwych zalet wymienia się odpowiedź na kognitywistyczne słabości diagnostów, którzy poddani są ludzkim odruchom, zmęczeniom i rozproszeniom. Przykładowo zauważa się problemy w znajdowaniu zmian wynikłych z trudnością utrzymania koncentracji w badaniach radiologicznych, gdyż występują one na tle stosunkowo rzadko\cite{CADPrzelaskowski2010}. Ponadto zaletą jest stosowanie w większości technik komputerowych metod ilościowych w detekcji cech, na podstawie których stwierdza się wystąpienie zmiany bądź jej braku. Dzięki temu jasno określona metryka, a nie subiektywny odbiór analityka, jest podłożem do wystawienia diagnozy.
\par
Przypuszcza się, że popyt na powyższe techniki będzie rósł wraz z ogólnym trendem wzrostu zapotrzebowania na usługi medyczne, spowodowanym zmianami demograficznymi w starzejących się społeczeństwach Europy oraz Azji, a kurczący się rynek pracy dodatkowo będzie podbijać koszty, jakie ponosić będzie służba zdrowia. Trend w zastosowaniu technik komputerowych w opiece medycznej uważa się za immanentną rzeczywistość.
\par
Nieprzyjęcie bardziej zaawansowanych technik sztucznej inteligencji w medycynie może być związane z olbrzymim kosztem alternatywnym, jaki już wiązał się z wdrożeniem dużych generatywnych sieci neuronowych oraz implementacją globalnych systemów. Koszt ten, w opinii autora, w pewnym stopniu zmusza inwestorów do wsparcia rozwoju wyżej wspomnianych technik.
\par 
Niniejsza praca ma za zadanie przedstawić narzędzie, będące w istocie wytrenowaną, splotową siecią neuronową, służące detekcji i segmentacji lezji stwardnienia rozsianego MS \la{multiple sclerosis} na obrazie MRI przekroju mózgu. Czynność ta dziś jest ręcznie wykonywana przez fizycznych diagnostów, a klasyczne komputerowe metody segmentacyjne nie dawały rezultatów ze względu na charakter lezji na obrazie. Samo uproszczenie procesu diagnozy MS, w razie wysokiej efektywności metody może spowodować wymierną redukcję kosztu oraz skrócenie czasu jej przeprowadzenia. Praca ma wstępnie rozstrzygnąć czy narzędzie będzie wpisywało się w nurt CAD czy raczej w pełni zautomatyzowanej diagnostyki obrazowej. 
\par
Kolejnym aspektem niniejszej pracy jest analiza porównawcza doborów funkcji kosztu, optymalizatora oraz modalności obrazu w kontekście jakości przedstawianego rozwiązania. Analiza ta jest przeprowadzana na podstawie metryk podobieństwa obrazów, których wartości otrzymywane są po przepuszczeniu zbioru ewaluacyjnego przez w pełni wytrenowaną sieć, oraz przez osobistą ocenę autora pokrywania się wyjścia sieci z wartością oczekiwaną. Z oczywistych względów jest to ocena obarczona błędem i winna być skorygowana przez zawodowego diagnostę. Celem zrozumienia rezultatów przedstawia się szeroki przegląd tematyki sieci neuronowych, wraz ze szczegółowym opisem wykorzystanego modelu, funkcji kosztu oraz metod optymalizacyjnych.
\par 
Narzędzie przygotowane w ramach pracy ma zastosowanie w diagnostyce stwardnienia rozsianego. Samo MS jest przewlekłą chorobą degradacji ośrodkowego układu nerwowego, której etiologia wciąż nie jest znana nauce\cite{ZEPHIR2018358} i jest jedną z najczęstszych przyczyn niepełnosprawności u ludzi młodych. Występuje u od 2 do 150 na 100 000 ludności w zależności od kraju i konkretnej populacji\cite{Rosati2001}. Nie ma żadnych leków które potrafiłyby wyleczyć z choroby, stosuje się jedynie takie, które wstrzymują bądź spowalniają jej rozwój. Koszty leczenia dla służby zdrowia są duże. Opakowanie leku Tecfidera 240 mg po 56 kapsułek kosztuje w hurcie około pięciu tysięcy złotych brutto, a starcza pacjentowi zazwyczaj zaledwie na miesiąc. Ponadto późne wykrycie, skutkujące niepełnosprawnością, generuje koszty dla segmentu ubezpieczeń społecznych, ze względu na ograniczenie bądź niemożność podjęcia pracy zarobkowej. Diagnoza na wczesnym etapie pozwala oszczędzić służbie zdrowia i pacjentom wymierne środki.
\par
Wyzwaniem dla proponowanego rozwiązania jest wysoka precyzja, dokładność, pozwalająca na zastąpienie diagnosty w procesie segmentacji lezji MS. 


% \par
% Wraz ze zmianami demograficznymi w starzejących się społeczeństwach Europy oraz Azji, przypuszcza się, że popyt na usługi medyczne się zwiększy. Spowodują one zwiększenie kosztów pracodawcy będące rezultatem zmniejszającego się rynku pracy. To wszystko powoduje, że jednym z najważniejszych paradygmatów służby zdrowia jest uczynienie procesów bardziej czasowo oraz kosztowo efektywnymi, na co niebagatelny wpływ może mieć ich automatyzacja i uproszczenie, które to osiąga się między innymi przez wykorzystanie technik komputerowych, w tym sztucznej inteligencji. Temat niniejszej pracy wpisuje się więc w wielki wysiłek naukowy mający na celu odpowiedzieć na wyzwania dzisiejszych czasów. 

  

% \par
% Współczesny rozwój medycyny w dużej mierze definiowany jest rozwojem techniki, dającym między innymi coraz to efektywniejsze metody akwizycji danych oraz ich analizy, dostarczającym nowocześniejsze narzędzia lekarzom oraz usprawniającym procesy administracyjne. Współcześnie standardowym procesem w diagnostyce dużej części nowotworów jest wykorzystanie technik obrazowania medycznego opartych na promieniowaniu jonizującym oraz  ultrasonograficznych, a do mierzenia parametrów życiowych stosuje się urządzenia takie jak pulsoksymetry, glukometry, ciśnieniomierze i wiele innych. Są to rozwiązania i urządzenia dostępne na rynku od dłuższego czasu, wymiernie wspierające segment zdrowia publicznego i przekładające się na dłuższą przeżywalność osób objętych opieką zdrowotną.

% Rozwój zautomatyzowanych technik CAD, mimo wielu lat badań, jest hamowany przez często niską  
% \par
% Rozwój technik CAD nie odbywał się bez problemów. Według prof. dr. hab. Artura Przelaskowskiego sama schematyzacja procesu diagnozy jest trudna, jako że "istnieje wiele szkół, indywidualnych metod i sposobów, nawyków i intuicyjnych przyzwyczajeń, a próby standaryzacji napotykają spory opór"\cite{CADPrzelaskowski2010}. Jest to rzecz którą brać należy pod uwagę w dowolnej pracy z zakresu komputerowo wspomaganej diagnostyki medycznej. Co jeszcze przyznaje, techniki komputerowe odpowiadają na niektóre słabości kognitywistyczne lekarzy i analityków. Wiele zmian i anormalności, będących podłożem do rozpoznania choroby występuje w niewielkim stopniu w zestawieniu z tłem, a dla analityka, mającego za zadanie detekcję tej małej ilości zmian w dużej ilości danych, utrzymanie koncentracji staje się dodatkowym obciążeniem. Jest to widoczne w badaniach radiologicznych, w których zmiany rakowe najczęściej dotyczą od 3 do 4 na 1000 komórek. Metody komputerowe wykorzystują metody ilościowe w detekcji cech na podstawie którego stwierdzają wystąpienie zmiany bądź jej braku, dzięki czemu jasno określona metryka, a nie subiektywny odbiór analityka jest podłożem do wystawienia diagnozy. 

% \par 
% Mając to wszystko na uwadze, praca niniejsza jest swoistą odpowiedzią na problematykę konkretnego problemu z dziedziny CAD, jakim jest wspomożenie lekarza w procesie diagnozy Stwardnienia Rozsianego \la{sclerosis multiplex}. Proces ten zakłada zastosowanie kryteriów MacDonalda, do których zaliczana jest analiza obrazu mózgu tomografii rezonansu magnetycznego MRI celem znalezienia zmian demielizacyjnych, czyli lezji. 

% Jest to przewlekła choroba ośrodkowego układu nerwowego dotykająca w Europie od 4 do 193 na 100 000 ludności\cite{PUGLIATTI2002182}, i często do niepełnosprawności. 