\section{Testy i opis wyników}
\label{sec:Tests}
\subsection{Wstęp}

Wytrenowano trzy sieci neuronowe z podanymi parametrami:
\begin{table}[H]
	\centering
	\caption{Parametry uczonych sieci U-Net}
	\begin{tabular}{|c|c|c|c|}
		\hline
		\textbf{} & \textbf{BCE-20} & \textbf{BCE-400} & \textbf{FL-2} \\
		\hline
		\textbf{Optymalizator} & Adam & Adam & Adam \\
		\hline
		\textbf{Funkcja Kosztu} & BCE & BCE & FL \\
		\hline
		\textbf{Ważenie} & 20 & 400 & 2 \\
		\hline
		\textbf{Rozmiar serii (batch)} & 8 & 8 & 8 \\
		\hline
		\textbf{Szybkość uczenia} &1e-4 & 1e-4&1e-4\\
		\hline
		\textbf{Epoki} & 50  & 50 & 50 \\
		\hline
	\end{tabular}
	\label{tab:mytable}
\end{table}
\par
Wszystkie sieci trenowano obrazami wielkości 128x128, rozdzielono zbiór uczący od ewaluacyjnego stosunkiem 4 do 1. Wszystkie zbiory uczące mieszano w procesie uczącym. Zastosowano wszędzie skaler \texttt{GradScaler}. Ważenie odnosi się do klasy mniejszościowej zbioru, tj. przypadków klasyfikacji lezji w BCE-20 i BCE-400. W przypadku FL-2 oznacza ono parametr $\gamma$ funkcji kosztu. Przypomina się, że sumaryczny stosunek pikseli lezji do pikseli tła w zbiorze danych wynosi 1:400, toteż w uczeniu sieci BCE-400 kompletnie balansuje się funkcję kosztu, w przypadku BCE-20 jest ona częściowo zbalansowana. 

\par
Co warto napomnieć, wartość współczynnika F1 obliczana została ze wzoru  $ \mathrm{F_1 = \frac{2\cdot{}|TP|}{2|TP|+|FP|+|FN|}}$ a nie  $\mathrm{2\frac{PPV\cdot{}TPR}{PPV+TPR}}$ W przypadku mianownika równego zero wartość metryki ustalana była arbitralnie na zero, co jest praktyką wykorzystywaną przez większość bibliotek obliczających tą metrykę jak TorchMetrics. Ponieważ mianowniki w obu tych wzorach nie zawsze równocześnie wynoszą zero, to wartości metryk obliczane z obu tych wzorów minimalnie się różnią. Tym samym, obliczona wartość współczynnika Sørensen'a Dice'a nie jest równa średniej harmonicznej z czułości i precyzji. Wykorzystując nauczone modele BCE-20 oraz BCE-400, najlepsze wyniki metryk otrzymano dla progów przypisania do klas wynoszącym 0.9, w przypadku modelu FL-2 najlepsze otrzymano dla progu 0.5.

\subsection{Wytrenowane sieci}
\label{sec:trained-nets}


\subsubsection{BCE-20}

\par
Sieć BCE-20 wykazuje się dobrymi wynikami współczynnika Sørensen'a-Dice'a oraz Jaccarda, oraz najlepsze wyniki współczynnika Kappa spośród wytrenowanych sieci. Uczenie sieci przebiegło w sposób stabilny. Wskaźniki czułości i precyzji również są na wysokim poziomie, osiągając wartość 0.748 oraz 0.912 w ostatniej epoce uczenia.

%\addImage{fig:bce20-2297}{src/img/Tests/bce20-2297.png}{BCE-20 trzy małe lezje}
%\addImage{fig:bce20-4556}{src/img/Tests/bce20-4556.png}{BCE-20 większe lezje}
%\addImage{fig:bce20-9482}{src/img/Tests/bce20-9482.png}{BCE-20 wiele lezji różnych wielkości}





\begin{longtable}[p]{|c|c|c|c|c|c|c|c|c|c|c|c|}
	\caption{Metryki BCE-20} \\
	
	\hline
	\textbf{Epoch} & \textbf{F1} & \textbf{JACC} & \textbf{PPV} & \textbf{TPR} & \textbf{TNR} & \textbf{BACC} & \textbf{K} & \textbf{F05} & \textbf{F2} & \textbf{MCC} \\
	\hline
	\endfirsthead
	
	\multicolumn{11}{c}%
	{{\tablename\ \thetable{} -- Kontynuacja z poprzedniej strony}} \\
	\hline
	\textbf{Epoch} & \textbf{F1} & \textbf{JACC} & \textbf{PPV} & \textbf{TPR} & \textbf{TNR} & \textbf{BACC} & \textbf{K} & \textbf{F05} & \textbf{F2} & \textbf{MCC} \\
	\hline
	\endhead
	
	\hline \multicolumn{11}{|r|}{{Kontynuacja na następnej stronie}} \\ \hline
	\endfoot
	
	\hline
	\endlastfoot
	
	
	1 & 0.612 & 0.459 & 0.517 & 0.781 & 0.998 & 0.880 & 0.477 & 0.569 & 0.775 & 0.601 \\
	2 & 0.652 & 0.501 & 0.557 & 0.809 & 0.998 & 0.894 & 0.523 & 0.624 & 0.792 & 0.659 \\
	3 & 0.639 & 0.495 & 0.822 & 0.546 & 0.999 & 0.763 & 0.488 & 0.586 & 0.757 & 0.568 \\
	4 & 0.707 & 0.564 & 0.640 & 0.805 & 0.999 & 0.892 & 0.591 & 0.682 & 0.833 & 0.688 \\
	5 & 0.726 & 0.589 & 0.734 & 0.731 & 0.999 & 0.855 & 0.647 & 0.735 & 0.846 & 0.722 \\
	6 & 0.730 & 0.593 & 0.774 & 0.702 & 0.999 & 0.841 & 0.661 & 0.748 & 0.855 & 0.736 \\
	7 & 0.741 & 0.607 & 0.722 & 0.770 & 0.999 & 0.875 & 0.686 & 0.771 & 0.868 & 0.772 \\
	8 & 0.736 & 0.601 & 0.785 & 0.707 & 0.999 & 0.843 & 0.664 & 0.752 & 0.857 & 0.746 \\
	9 & 0.752 & 0.618 & 0.711 & 0.809 & 0.999 & 0.894 & 0.703 & 0.790 & 0.891 & 0.791 \\
	10 & 0.750 & 0.617 & 0.703 & 0.814 & 0.999 & 0.897 & 0.703 & 0.791 & 0.890 & 0.791 \\
	11 & 0.759 & 0.628 & 0.716 & 0.815 & 0.999 & 0.897 & 0.720 & 0.806 & 0.900 & 0.806 \\
	12 & 0.762 & 0.631 & 0.711 & 0.829 & 0.999 & 0.904 & 0.731 & 0.816 & 0.909 & 0.819 \\
	13 & 0.767 & 0.637 & 0.706 & 0.848 & 0.999 & 0.914 & 0.750 & 0.834 & 0.923 & 0.839 \\
	14 & 0.766 & 0.637 & 0.702 & 0.853 & 0.999 & 0.916 & 0.752 & 0.835 & 0.924 & 0.841 \\
	15 & 0.769 & 0.641 & 0.718 & 0.837 & 0.999 & 0.908 & 0.753 & 0.835 & 0.924 & 0.841 \\
	16 & 0.773 & 0.645 & 0.715 & 0.848 & 0.999 & 0.914 & 0.768 & 0.852 & 0.935 & 0.857 \\
	17 & 0.778 & 0.653 & 0.731 & 0.840 & 0.999 & 0.910 & 0.775 & 0.860 & 0.939 & 0.864 \\
	18 & 0.769 & 0.642 & 0.704 & 0.859 & 0.999 & 0.919 & 0.769 & 0.853 & 0.936 & 0.858 \\
	19 & 0.777 & 0.652 & 0.721 & 0.849 & 0.999 & 0.914 & 0.788 & 0.872 & 0.947 & 0.876 \\
	20 & 0.770 & 0.642 & 0.741 & 0.809 & 0.999 & 0.907 & 0.769 & 0.853 & 0.937 & 0.859 \\
	21 & 0.777 & 0.671 & 0.739 & 0.835 & 0.999 & 0.917 & 0.790 & 0.872 & 0.947 & 0.875 \\
	22 & 0.791 & 0.670 & 0.756 & 0.853 & 0.999 & 0.926 & 0.815 & 0.896 & 0.957 & 0.895 \\
	23 & 0.781 & 0.664 & 0.755 & 0.844 & 0.999 & 0.921 & 0.798 & 0.883 & 0.952 & 0.881 \\
	24 & 0.787 & 0.674 & 0.726 & 0.854 & 0.999 & 0.927 & 0.811 & 0.892 & 0.956 & 0.893 \\
	25 & 0.793 & 0.674 & 0.737 & 0.850 & 0.999 & 0.925 & 0.823 & 0.903 & 0.961 & 0.906 \\
	26 & 0.790 & 0.678 & 0.772 & 0.727 & 0.999 & 0.907 & 0.815 & 0.894 & 0.957 & 0.897 \\
	27 & 0.792 & 0.679 & 0.744 & 0.855 & 0.999 & 0.927 & 0.827 & 0.905 & 0.961 & 0.908 \\
	28 & 0.792 & 0.678 & 0.750 & 0.850 & 0.999 & 0.925 & 0.825 & 0.903 & 0.961 & 0.907 \\
	29 & 0.791 & 0.679 & 0.765 & 0.834 & 0.999 & 0.917 & 0.825 & 0.903 & 0.961 & 0.907 \\
	30 & 0.793 & 0.683 & 0.748 & 0.860 & 0.999 & 0.929 & 0.831 & 0.909 & 0.964 & 0.915 \\
	31 & 0.795 & 0.686 & 0.746 & 0.863 & 0.999 & 0.931 & 0.836 & 0.914 & 0.966 & 0.920 \\
	32 & 0.796 & 0.686 & 0.756 & 0.855 & 0.999 & 0.927 & 0.834 & 0.912 & 0.965 & 0.918 \\
	33 & 0.800 & 0.696 & 0.748 & 0.871 & 0.999 & 0.935 & 0.848 & 0.926 & 0.971 & 0.933 \\
	34 & 0.797 & 0.693 & 0.765 & 0.847 & 0.999 & 0.923 & 0.846 & 0.924 & 0.970 & 0.930 \\
	35 & 0.799 & 0.695 & 0.748 & 0.873 & 0.999 & 0.936 & 0.850 & 0.928 & 0.972 & 0.935 \\
	36 & 0.797 & 0.694 & 0.747 & 0.877 & 0.999 & 0.938 & 0.848 & 0.926 & 0.971 & 0.933 \\
	37 & 0.799 & 0.697 & 0.748 & 0.880 & 0.999 & 0.939 & 0.853 & 0.931 & 0.973 & 0.938 \\
	38 & 0.799 & 0.698 & 0.743 & 0.889 & 0.999 & 0.944 & 0.855 & 0.933 & 0.974 & 0.940 \\
	39 & 0.800 & 0.700 & 0.747 & 0.885 & 0.999 & 0.942 & 0.858 & 0.936 & 0.975 & 0.943 \\
	40 & 0.802 & 0.702 & 0.748 & 0.889 & 0.999 & 0.944 & 0.863 & 0.941 & 0.977 & 0.947 \\
	41 & 0.803 & 0.704 & 0.748 & 0.895 & 0.999 & 0.947 & 0.868 & 0.946 & 0.978 & 0.952 \\
	42 & 0.804 & 0.705 & 0.748 & 0.899 & 0.999 & 0.949 & 0.871 & 0.949 & 0.979 & 0.955 \\
	43 & 0.805 & 0.706 & 0.748 & 0.901 & 0.999 & 0.950 & 0.874 & 0.952 & 0.980 & 0.958 \\
	44 & 0.806 & 0.708 & 0.748 & 0.904 & 0.999 & 0.952 & 0.878 & 0.956 & 0.981 & 0.961 \\
	45 & 0.808 & 0.709 & 0.748 & 0.908 & 0.999 & 0.954 & 0.882 & 0.960 & 0.982 & 0.965 \\
	46 & 0.809 & 0.710 & 0.748 & 0.912 & 0.999 & 0.956 & 0.885 & 0.963 & 0.983 & 0.968 \\
	47 & 0.810 & 0.712 & 0.748 & 0.916 & 0.999 & 0.958 & 0.889 & 0.967 & 0.984 & 0.972 \\
	48 & 0.811 & 0.713 & 0.748 & 0.920 & 0.999 & 0.960 & 0.892 & 0.970 & 0.985 & 0.975 \\
	49 & 0.812 & 0.715 & 0.748 & 0.924 & 0.999 & 0.962 & 0.896 & 0.974 & 0.986 & 0.979 \\
	50 & 0.812 & 0.717 & 0.748 & 0.928 & 0.999 & 0.964 & 0.899 & 0.977 & 0.987 & 0.982 
	\label{tab:metrics-bce20}
\end{longtable}
\addImage{fig:bce20}{src/img/Tests/bce20.png}{BCE-20 przykładowe obrazy}
\subsubsection{BCE-400}
\par
Sieć BCE-400 wykazuje się najgorszymi metrykami z trzech przygotowanych sieci. Współczynnik Sørensen'a-Dice'a wyniósł po 50 epokach 0.661, przy 0.812 w sieci BCE-20. Na wyjściach sieci widać, że wiele jest na niej fałszywie pozytywnych wskazań, wyznacza ona wiele lezji których nie ma na obrazie odniesienia. Dobrze przedstawia to rysunek \ref{fig:bce400}. Wartość precyzji,wynosi zaledwie 0.515 po 50 epokach uczenia. Proces uczenia nie przebiegał w sposób stabilny, a metryki zmieniały się chaotycznie.

%\addImage{fig:bce400-2297}{src/img/Tests/bce400-2297.png}{BCE-400 trzy małe lezje}
%\addImage{fig:bce400-4556}{src/img/Tests/bce400-4556.png}{BCE-400 większe lezje}
%\addImage{fig:bce400-9482}{src/img/Tests/bce400-9482.png}{BCE-400 wiele lezji różnych wielkości}

\addImage{fig:bce400}{src/img/Tests/bce400.png}{BCE-400 przykładowe obrazy}
\begin{longtable}[p]{|c|c|c|c|c|c|c|c|c|c|c|}
	
	\caption{Metryki BCE-400} \\
	\hline
	\textbf{Epoch} & \textbf{F1} & \textbf{JACC} & \textbf{PPV} & \textbf{TPR} & \textbf{TNR} & \textbf{BACC} & \textbf{K} & \textbf{F05} & \textbf{F2} & \textbf{MCC} \\
	\hline
	\endfirsthead
	
	\multicolumn{11}{c}%
	{{\tablename\ \thetable{} -- Kontynuacja z poprzedniej strony}} \\
	\hline
	\textbf{Epoch} & \textbf{F1} & \textbf{JACC} & \textbf{PPV} & \textbf{TPR} & \textbf{TNR} & \textbf{BACC} & \textbf{K} & \textbf{F05} & \textbf{F2} & \textbf{MCC} \\
	\hline
	\endhead
	
	\hline \multicolumn{11}{|r|}{{Kontynuacja na następnej stronie}} \\ \hline
	\endfoot
	
	\hline
	\endlastfoot
	
	1 & 0.311 & 0.191 & 0.193 & 0.932 & 0.991 & 0.957 & 0.308 & 0.281 & 0.348 & 0.409 \\
	2 & 0.363 & 0.229 & 0.232 & 0.943 & 0.992 & 0.963 & 0.360 & 0.330 & 0.403 & 0.455 \\
	3 & 0.328 & 0.204 & 0.206 & 0.952 & 0.991 & 0.967 & 0.325 & 0.297 & 0.366 & 0.426 \\
	4 & 0.431 & 0.284 & 0.286 & 0.953 & 0.994 & 0.969 & 0.429 & 0.396 & 0.472 & 0.511 \\
	5 & 0.496 & 0.339 & 0.344 & 0.942 & 0.996 & 0.964 & 0.494 & 0.461 & 0.537 & 0.561 \\
	6 & 0.440 & 0.290 & 0.292 & 0.956 & 0.994 & 0.971 & 0.438 & 0.405 & 0.483 & 0.520 \\
	7 & 0.574 & 0.414 & 0.424 & 0.929 & 0.997 & 0.958 & 0.573 & 0.541 & 0.612 & 0.621 \\
	8 & 0.525 & 0.365 & 0.370 & 0.950 & 0.996 & 0.969 & 0.524 & 0.490 & 0.567 & 0.586 \\
	9 & 0.490 & 0.332 & 0.336 & 0.955 & 0.995 & 0.971 & 0.488 & 0.454 & 0.533 & 0.559 \\
	10 & 0.504 & 0.346 & 0.349 & 0.961 & 0.996 & 0.974 & 0.504 & 0.468 & 0.546 & 0.571 \\
	11 & 0.505 & 0.347 & 0.350 & 0.963 & 0.996 & 0.975 & 0.504 & 0.469 & 0.547 & 0.572 \\
	12 & 0.483 & 0.327 & 0.329 & 0.967 & 0.995 & 0.977 & 0.481 & 0.447 & 0.526 & 0.556 \\
	13 & 0.502 & 0.344 & 0.346 & 0.963 & 0.996 & 0.974 & 0.501 & 0.466 & 0.545 & 0.571 \\
	14 & 0.569 & 0.408 & 0.415 & 0.939 & 0.997 & 0.968 & 0.568 & 0.535 & 0.608 & 0.618 \\
	15 & 0.542 & 0.382 & 0.386 & 0.960 & 0.996 & 0.973 & 0.542 & 0.507 & 0.584 & 0.602 \\
	16 & 0.530 & 0.372 & 0.377 & 0.955 & 0.996 & 0.976 & 0.530 & 0.496 & 0.571 & 0.590 \\
	17 & 0.545 & 0.384 & 0.389 & 0.960 & 0.996 & 0.978 & 0.545 & 0.511 & 0.586 & 0.603 \\
	18 & 0.535 & 0.374 & 0.378 & 0.961 & 0.996 & 0.977 & 0.535 & 0.501 & 0.576 & 0.593 \\
	19 & 0.578 & 0.416 & 0.420 & 0.953 & 0.996 & 0.975 & 0.578 & 0.544 & 0.620 & 0.628 \\
	20 & 0.584 & 0.422 & 0.427 & 0.956 & 0.997 & 0.978 & 0.584 & 0.550 & 0.626 & 0.634 \\
	21 & 0.546 & 0.384 & 0.389 & 0.962 & 0.997 & 0.978 & 0.546 & 0.511 & 0.586 & 0.604 \\
	22 & 0.605 & 0.444 & 0.448 & 0.606 & 0.997 & 0.802 & 0.604 & 0.571 & 0.643 & 0.665 \\
	23 & 0.554 & 0.391 & 0.394 & 0.576 & 0.997 & 0.787 & 0.553 & 0.521 & 0.592 & 0.617 \\
	24 & 0.606 & 0.425 & 0.432 & 0.625 & 0.997 & 0.811 & 0.606 & 0.575 & 0.644 & 0.668 \\
	25 & 0.552 & 0.444 & 0.439 & 0.552 & 0.997 & 0.774 & 0.552 & 0.521 & 0.595 & 0.614 \\
	26 & 0.586 & 0.391 & 0.395 & 0.586 & 0.997 & 0.791 & 0.586 & 0.554 & 0.625 & 0.647 \\
	27 & 0.572 & 0.424 & 0.429 & 0.575 & 0.997 & 0.786 & 0.572 & 0.538 & 0.611 & 0.631 \\
	28 & 0.571 & 0.410 & 0.416 & 0.613 & 0.997 & 0.805 & 0.571 & 0.537 & 0.609 & 0.630 \\
	29 & 0.600 & 0.431 & 0.436 & 0.626 & 0.997 & 0.812 & 0.600 & 0.566 & 0.637 & 0.662 \\
	30 & 0.623 & 0.413 & 0.420 & 0.613 & 0.997 & 0.805 & 0.623 & 0.590 & 0.660 & 0.686 \\
	31 & 0.594 & 0.441 & 0.447 & 0.596 & 0.997 & 0.796 & 0.594 & 0.560 & 0.630 & 0.656 \\
	32 & 0.547 & 0.434 & 0.433 & 0.543 & 0.997 & 0.767 & 0.547 & 0.513 & 0.588 & 0.610 \\
	33 & 0.606 & 0.459 & 0.467 & 0.491 & 0.997 & 0.744 & 0.606 & 0.573 & 0.644 & 0.671 \\
	34 & 0.627 & 0.465 & 0.471 & 0.542 & 0.997 & 0.769 & 0.627 & 0.594 & 0.666 & 0.696 \\
	35 & 0.616 & 0.448 & 0.454 & 0.554 & 0.997 & 0.776 & 0.616 & 0.582 & 0.655 & 0.684 \\
	36 & 0.626 & 0.439 & 0.445 & 0.576 & 0.997 & 0.786 & 0.626 & 0.593 & 0.665 & 0.696 \\
	37 & 0.651 & 0.463 & 0.468 & 0.628 & 0.997 & 0.813 & 0.651 & 0.618 & 0.691 & 0.719 \\
	38 & 0.623 & 0.464 & 0.470 & 0.616 & 0.997 & 0.807 & 0.623 & 0.589 & 0.661 & 0.687 \\
	39 & 0.634 & 0.472 & 0.479 & 0.624 & 0.997 & 0.811 & 0.634 & 0.602 & 0.672 & 0.698 \\
	40 & 0.651 & 0.466 & 0.473 & 0.633 & 0.997 & 0.811 & 0.651 & 0.618 & 0.690 & 0.720 \\
	41 & 0.612 & 0.431 & 0.435 & 0.608 & 0.997 & 0.802 & 0.612 & 0.575 & 0.650 & 0.669 \\
	42 & 0.638 & 0.438 & 0.442 & 0.639 & 0.997 & 0.818 & 0.638 & 0.603 & 0.676 & 0.700 \\
	43 & 0.638 & 0.444 & 0.448 & 0.644 & 0.997 & 0.821 & 0.638 & 0.603 & 0.675 & 0.698 \\
	44 & 0.657 & 0.451 & 0.456 & 0.657 & 0.997 & 0.827 & 0.657 & 0.622 & 0.696 & 0.719 \\
	45 & 0.613 & 0.450 & 0.457 & 0.612 & 0.997 & 0.805 & 0.613 & 0.577 & 0.651 & 0.670 \\
	46 & 0.648 & 0.488 & 0.499 & 0.648 & 0.998 & 0.823 & 0.648 & 0.610 & 0.688 & 0.710 \\
	47 & 0.654 & 0.496 & 0.508 & 0.653 & 0.998 & 0.825 & 0.654 & 0.617 & 0.694 & 0.714 \\
	48 & 0.641 & 0.483 & 0.494 & 0.640 & 0.998 & 0.822 & 0.641 & 0.602 & 0.681 & 0.702 \\
	49 & 0.646 & 0.482 & 0.494 & 0.647 & 0.998 & 0.823 & 0.646 & 0.607 & 0.685 & 0.706 \\
	50 & 0.661 & 0.504 & 0.515 & 0.651 & 0.998 & 0.828 & 0.661 & 0.622 & 0.696 & 0.719
	\label{tab:metrics-bce-400}
\end{longtable}


\subsubsection{FL-2}
Sieć FL-2 również charakteryzuje się dobrymi wynikami. Współczynnik Sørensen'a-Dice'a wynoszący 0.836 jest najwyższy spośród trenowanych sieci. Sieć odznacza się również dobrą precyzją wynoszącą 0.865 oraz dobrą czułością i współczynnikiem Kappa. W procesie uczenia sieć w miarę szybko i płynnie zaczęła otrzymywać dobre wyniki ewaluacji. 



%\addImage{fig:fl2-2297}{src/img/Tests/fl2-2297.png}{FL-2 trzy małe lezje}
%\addImage{fig:fl2-4556}{src/img/Tests/fl2-4556.png}{FL-2 większe lezje}
%\addImage{fig:fl2-9482}{src/img/Tests/fl2-9482.png}{FL-2 wiele lezji różnych wielkości}



\begin{longtable}[p]{|c|c|c|c|c|c|c|c|c|c|c|}
	\caption{Metryki FL-2} \\
	\hline
	\textbf{Epoch} & \textbf{F1} & \textbf{JACC} & \textbf{PPV} & \textbf{TPR} & \textbf{TNR} & \textbf{BACC} & \textbf{K} & \textbf{F05} & \textbf{F2} & \textbf{MCC} \\
	\hline
	\endfirsthead
	
	\multicolumn{11}{c}%
	{{\tablename\ \thetable{} -- Kontynuacja z poprzedniej strony}} \\
	\hline
	\textbf{Epoch} & \textbf{F1} & \textbf{JACC} & \textbf{PPV} & \textbf{TPR} & \textbf{TNR} & \textbf{BACC} & \textbf{K} & \textbf{F05} & \textbf{F2} & \textbf{MCC} \\
	\hline
	\endhead
		
	\hline \multicolumn{11}{|r|}{{Kontynuacja na następnej stronie}} \\ \hline

	\endfoot
	% \hline \multicolumn{11}{|r|}{{Kontynuacja na następnej stronie}} \\ \hline
	\hline
	\endlastfoot
	
	1 & 0.640 & 0.490 & 0.725 & 0.595 & 0.999 & 0.793 & 0.639 & 0.651 & 0.630 & 0.649\\
	2 & 0.695 & 0.552 & 0.723 & 0.688 & 0.999 & 0.839 & 0.694 & 0.698 & 0.692 & 0.699 \\
	3 & 0.704 & 0.564 & 0.722 & 0.706 & 0.999 & 0.848 & 0.704 & 0.706 & 0.703 & 0.708\\
	4 & 0.704 & 0.563 & 0.815 & 0.636 & 0.999 & 0.813 & 0.703 & 0.718 & 0.690 & 0.713\\
	5 & 0.723 & 0.584 & 0.813 & 0.664 & 0.999 & 0.827 & 0.723 & 0.736 & 0.711 & 0.730\\
	6 & 0.689 & 0.547 & 0.863 & 0.589 & 0.999 & 0.790 & 0.689 & 0.712 & 0.669 & 0.706\\
	7 & 0.753 & 0.621 & 0.804 & 0.720 & 0.999 & 0.855 & 0.752 & 0.760 & 0.747 & 0.757\\
	8 & 0.724 & 0.587 & 0.851 & 0.644 & 0.999 & 0.817 & 0.724 & 0.741 & 0.708 & 0.735 \\
	9 & 0.760 & 0.629 & 0.794 & 0.739 & 0.999 & 0.865 & 0.760 & 0.765 & 0.756 & 0.763\\
	10 & 0.766 & 0.636 & 0.852 & 0.703 & 0.999 & 0.847 & 0.766 & 0.779 & 0.754 & 0.771\\
	11 & 0.782 & 0.656 & 0.838 & 0.741 & 0.999 & 0.865 & 0.782 & 0.790 & 0.774 & 0.785\\
	12 & 0.777 & 0.649 & 0.858 & 0.718 & 0.999 & 0.854 & 0.777 & 0.788 & 0.765 & 0.782\\
	13 & 0.786 & 0.662 & 0.846 & 0.743 & 0.999 & 0.867 & 0.786 & 0.795 & 0.778 & 0.790\\
	14 & 0.785 & 0.661 & 0.872 & 0.722 & 0.999 & 0.856 & 0.785 & 0.797 & 0.773 & 0.790\\
	15 & 0.797 & 0.677 & 0.858 & 0.753 & 0.999 & 0.872 & 0.797 & 0.806 & 0.788 & 0.801\\
	16 & 0.804 & 0.686 & 0.853 & 0.767 & 0.999 & 0.879 & 0.804 & 0.812 & 0.798 & 0.807\\
	17 & 0.808 & 0.692 & 0.853 & 0.773 & 0.999 & 0.882 & 0.808 & 0.815 & 0.801 & 0.810\\
	18 & 0.808 & 0.688 & 0.868 & 0.760 & 0.999 & 0.882 & 0.808 & 0.815 & 0.796 & 0.809\\
	19 & 0.805 & 0.688 & 0.854 & 0.805 & 0.999 & 0.889 & 0.805 & 0.817 & 0.805 & 0.820\\
	20 & 0.818 & 0.706 & 0.869 & 0.819 & 0.999 & 0.889 & 0.818 & 0.825 & 0.812 & 0.822\\
	21 & 0.820 & 0.708 & 0.857 & 0.819 & 0.999 & 0.890 & 0.820 & 0.825 & 0.814 & 0.821\\
	22 & 0.819 & 0.706 & 0.859 & 0.822 & 0.999 & 0.889 & 0.819 & 0.818 & 0.813 & 0.825\\
	23 & 0.823 & 0.712 & 0.868 & 0.823 & 0.999 & 0.889 & 0.823 & 0.829 & 0.816 & 0.824\\
	24 & 0.822 & 0.711 & 0.855 & 0.828 & 0.999 & 0.894 & 0.822 & 0.827 & 0.817 & 0.826\\
	25 & 0.824 & 0.713 & 0.869 & 0.828 & 0.999 & 0.888 & 0.824 & 0.827 & 0.817 & 0.831\\
	26 & 0.830 & 0.722 & 0.837 & 0.830 & 0.999 & 0.895 & 0.830 & 0.831 & 0.829 & 0.831\\
	27 & 0.830 & 0.723 & 0.845 & 0.827 & 0.999 & 0.889 & 0.830 & 0.829 & 0.830 & 0.832\\
	28 & 0.830 & 0.723 & 0.857 & 0.831 & 0.999 & 0.891 & 0.831 & 0.830 & 0.830 & 0.832\\
	29 & 0.831 & 0.723 & 0.842 & 0.831 & 0.999 & 0.895 & 0.831 & 0.830 & 0.831 & 0.834\\
	30 & 0.831 & 0.724 & 0.854 & 0.830 & 0.999 & 0.891 & 0.831 & 0.832 & 0.832 & 0.835\\
	31 & 0.832 & 0.725 & 0.853 & 0.832 & 0.999 & 0.892 & 0.832 & 0.834 & 0.832 & 0.832\\
	32 & 0.834 & 0.728 & 0.827 & 0.834 & 0.999 & 0.893 & 0.834 & 0.837 & 0.835 & 0.828\\
	33 & 0.832 & 0.726 & 0.861 & 0.832 & 0.999 & 0.828 & 0.832 & 0.837 & 0.830 & 0.837\\
	34 & 0.824 & 0.727 & 0.726 & 0.833 & 0.999 & 0.837 & 0.824 & 0.830 & 0.826 & 0.834\\
	35 & 0.836 & 0.726 & 0.727 & 0.836 & 0.999 & 0.833 & 0.836 & 0.835 & 0.834 & 0.833\\
	36 & 0.832 & 0.727 & 0.731 & 0.833 & 0.999 & 0.846 & 0.832 & 0.836 & 0.830 & 0.833\\
	37 & 0.832 & 0.727 & 0.726 & 0.834 & 0.999 & 0.833 & 0.832 & 0.834 & 0.830 & 0.834\\
	38 & 0.832 & 0.727 & 0.853 & 0.834 & 0.999 & 0.839 & 0.832 & 0.834 & 0.832 &0.834 \\
	39 & 0.832 & 0.727 & 0.723 & 0.834 & 0.999 & 0.850 & 0.832 & 0.833 & 0.831& 0.834 \\
	40 & 0.833 & 0.727 & 0.731 & 0.834 & 0.999 & 0.841 & 0.833 & 0.835 & 0.831 & 0.835\\
	41 & 0.835 & 0.732 & 0.837 & 0.835 & 0.999 & 0.851 & 0.835 & 0.838 & 0.833 & 0.837\\
	42 & 0.832 & 0.731 & 0.827 & 0.832 & 0.999 & 0.890 & 0.832 & 0.832 & 0.830 & 0.835\\
	43 & 0.833 & 0.727 & 0.831 & 0.833 & 0.999 & 0.891 & 0.833 & 0.834 & 0.832 & 0.835\\
	44 & 0.834 & 0.725 & 0.835 & 0.834 & 0.999 & 0.909 & 0.834 & 0.835 & 0.834 & 0.832\\
	45 & 0.829 & 0.731 & 0.825 & 0.830 & 0.999 & 0.905 & 0.829 & 0.826 & 0.825 & 0.833\\
	46 & 0.836 & 0.732 & 0.827 & 0.836 & 0.999 & 0.902 & 0.836 & 0.838 & 0.827 & 0.838\\
	47 & 0.832 & 0.731 & 0.835 & 0.832 & 0.999 & 0.905 & 0.832 & 0.828 & 0.831 & 0.834\\
	48 & 0.836 & 0.731 & 0.835 & 0.837 & 0.999 & 0.890 & 0.836 & 0.829 & 0.829 & 0.837\\
	49 & 0.836 & 0.732 & 0.852 & 0.831 & 0.999 & 0.909 & 0.836 & 0.839 & 0.834 & 0.838\\
	50 & 0.836 & 0.732 & 0.865 & 0.830 & 0.999 & 0.893 & 0.836 & 0.839 & 0.835 &0.837

	\label{tab:metrics-bce-20}
\end{longtable}
\addImage{fig:fl2}{src/img/Tests/fl2.png}{FL-2 przykładowe obrazy}
\subsection{Porównanie z istniejącymi rozwiązaniami}

\par
Metryki wytrenowanych sieci przedstawionych w niniejszej pracy mogą wydawać się być lepszymi od metryk isteniejących rozwiązań. Wpływ na to mogły mieć różne parametry uczenia, w tym optymalizator Adam i ważenie funkcji kosztu. Zaznacza się jednak, że zbiór danych uczących  różni się ze zbiorami wykorzystywanymi w podobnych pracach, co powoduje pewną niemiarodajność w ocenie porównawczej wykonanego zadania na tle istniejących rozwiązań. W przypadku prac przedstawionych w rozdziale \ref{sec:MarketAndSolutionsOverview} wykorzystywany jest jedynie zbiór ISBI. Zauważa się, że proponowane sieci mają zdecydowanie lepsze parametry czułości w stosunku do wskaźnika F1. Porównanie zostało przedstawione w tabeli \ref{tab:metric-comparison}.

\begin{longtable}[p]{|c|c|c|c|}
	\caption{Porównanie wyników z istniejącymi rozwiązaniami} \\
	\hline
	\textbf{Metoda} & \textbf{Modalności} & \textbf{F1} & \textbf{TPR} \\
	\hline
	\endfirsthead
	
	\multicolumn{4}{c}%
	{{\tablename\ \thetable{} -- Kontynuacja z poprzedniej strony}} \\
	\hline
	\textbf{Metoda} & \textbf{Modalności} & \textbf{F1} & \textbf{TPR}  \\
	\hline
	\endhead
	
	\hline \multicolumn{4}{|r|}{{Kontynuacja na następnej stronie}} \\ \hline
	\endfoot
	
	\hline
	\endlastfoot
	
	AB-CNN & (FLAIR) & 0.632 & 0.455  \\
	ACU-NET & (T1, T2, FLAIR, PD) & 0.635 & 0.479  \\
	Asmsl & (T1, T2, FLAIR, PD) & 0.630 & 0.487  \\
	IMAGINE & (T1, T2, FLAIR, PD) & 0.584 & 0.456 \\
	Cascaded CNN & (T1, T2, FLAIR) & 0.630 & 0.367  \\
	DED-CNN & (T1, T2, FLAIR) & 0.486 & 0.303 \\
	\hline
	BCE-20 & (FLAIR) & 0.812 & 0.928\\
	BCE-400 & (FLAIR) & 0.661 & 0.651\\
	FL-2 & (FLAIR) & 0.836 & 0.830 
	\label{tab:metric-comparison}
\end{longtable}
